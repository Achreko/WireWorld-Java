\documentclass[11pt]{article}
\usepackage[polish]{babel}
\usepackage[T1]{fontenc}
\usepackage[utf8]{inputenc}
\usepackage{graphicx}
\usepackage{enumitem}

\begin{document}

\begin{huge}
\begin{center}
\textbf{Specyfikacja funkcjonalna}
\end{center}
\end{huge}

 \renewcommand{\labelenumii}{\Roman{enumii}}
 \begin{enumerate}
 
 \item Opis ogólny
 
 \begin{enumerate}[label=\arabic{enumi}.\arabic*.]
 
 \item Nazwa programu\\
 Program nazywa się $Wireworld$.
 \item Poruszany problem\\
 Implementować będziemy WireWorld Briana Silvermana.\\
 Komórka może znajdować się w jednym z czterech stanów:\\
 Pusta,\\
 Głowa elektronu,\\
 Ogon elektrony,\\
 Przewodnik.\\
 Zwykle przymuje się następujące kolory stanów: biały, czerwony, żółty, czarny.\\
 Kolejne generacje budowane są z wykorzystaniem zestawu pięciu zasad:\\
 Komórka pozostaje Pusta, jeśli była Pusta.\\
 Komórka staje się Ogonem elektronu, jeśli była Głową elektronu.\\
 Komórka staje się Przewodnikiem, jeśli była Ogonem elektronu.\\
 Komórka staje się Głową elektronu tylko wtedy, gdy dokładnie 1 lub 2 sąsiadujące komórki są Głowami Elektronu.\\
 Komórka staje się Przewodnikiem w każdym innym wypadku.\\
 W WireWorld stosuje się sąsiedztwo Moore'a.\\
 
 \item Użytkownik docelowy\\
 Użytkownikiem docelowym naszego programu jest prowadzący zajęć laboratoryjnych.\\

 \end{enumerate}
 
 
 
\item Ogólna funkcjonalność

\begin{enumerate}[label=\arabic{enumi}.\arabic*.]
 \item  Korzystanie z programu
 \\Program wykonany jest w formie aplikacji okienkowej. Znajduje się on w katalogu $automat komorkowy$ pod nazwą $Wireworld$. Występuje w formie pliku wykonywalnego o rozszerzeniu $jar.$ W tym samym folderze znajduje się plik $dane.txt$. Zawiera on dane na temat stanów wybranych komórek.\\
\item  Uruchamianie programu
\\Uruchomienie programu następuje przez klikniecie na plik „Wireworld”. Po tym program spyta ile generacji planszy użytkownik chce wykonać, a następnie wyświetli je po kolei.\\
\item Możliwosci programu
\\Program poza pokazaniem poszczególnych generacji planszy zapisuje również te generacje do folderu „generacje”. Przy każdym użyciu programu folder ten jest czyszczony i nowe generacje są zapisywane.\\
\end{enumerate}
 
  \item Format danych i struktura plików
  
  \begin{enumerate}[label=\arabic{enumi}.\arabic*.]
  
 \item Słownik\\
 *.png - nazwa pliku w postaci \textbf{gen-x.png}, gdzie $x$ jest numerem generacji,\\
 
 dane - umieszczone w jednym wierszu w pliku informacje o polach tabeli tworzącej nasz projekt, np. Diode: 0, 3, Normal w przypadku diod albo ElectronHead: 1, 3 w przypadku głów elektronu albo ElectronTali: 1, 9 w przypadku ogonów elektronu,\\
 
 GUI - graphical user interface, czyli to, co aplikacja pokazuje użytkownikowi,\\
 
 Macierz - tablica dwuwymiarowa o wymiarach X na Y, gdzie X to liczba wierszy, a Y to liczba kolumn,\\
 
 \item Struktura katalogów\\
 Plik \textbf{.jar} będzie przechowywany w katalogu \textbf{automat komórkowy} pod nazwą \textbf{Wireworld}. Dane znajdują się w tym samym katalogu. Dane wyjściowe przechowywane są w podkatalogu \textbf{generacje}.
 \item Przechowywanie danych w programie\\
 Nasz program znajduje się w repozytorium pod adresem $https://github.com/Achreko/WireWorld-Java$. Jeżeli zaś chodzi o struktury danych w programie, to planszę, na której będziemy umieszczać poszczególne elementy zapiszemy w postaci \textit{macierzy}. Będziemy też mieli drugą \textit{macierz}, w której znajdować się będzie przyszła generacja i to właśnie tam będziemy dokonywać zmian.\\
 \item Dane wejściowe\\
 Dane wejściowe znajdują się w pliku $dane.txt$. Zawiera on \textit{dane}.
 Jeśli użytkownik chce zmienić \textit{dane} lub wprowadzić nowe to musi edytować ten plik.\\
 \item Dane wyjściowe\\
 Dane wyjściowe będą w postaci plików \textit{*.png}, których ilość zależeć będzie od żądanej przez użytkownika ilości generacji.
 \end{enumerate}
 

\item Scenariusz działania programu


\begin{enumerate}[label=\arabic{enumi}.\arabic*.]
\item Scenariusz ogólny
\begin{enumerate}[label*=\arabic*.]
\item Program wczytuje \textit{dane} z pliku 
\item Program pyta użytkownika o liczbę generacji.
\item Program analizuje stany komórek i transformuje je.
\item Program pokazuje wynik pojedynczej generacji.
\item Program zapisuje pojedynczą generację do pliku.
\item Program ponownie wykonuje operacje z punktów 3-5, aż nie wykona ilości generacji równej tej 		podanej w punkcie 2.
\item Program pyta czy użytkownik chce jeszcze raz wykonać proces (powrót do punktu 1) dla tych 		samych danych, czy zakończyć.\\
\end{enumerate}



\item Scenariusz szczegółowy
\begin{enumerate}[label*=\arabic*.]
\item Program wczytuje \textit{dane} z pliku. 
\begin{itemize}
\item Program sprawdza czy sposób podania \textit{danych} jest poprawny, jeśli nie pokaże komunikat „błąd danych” i zakończy prace.
\end{itemize}
\item Program pyta użytkownika o liczbę generacji. 
\begin{itemize}
\item Sprawdzane jest czy użytkownik podał wartość numeryczną większą od 0. Jeśli tak to przechodzi dalej, jeśli nie to wyświetla informację o podaniu złej wartości i pyta ponownie o liczbę generacji.
\end{itemize}
\item Program usuwa wszystkie pliki z katalogu „generacje”.
\begin{itemize}
\item Jeśli katalog jest pusty to tylko idzie dalej, nic nie usuwa.
\end{itemize}
\item Program analizuje stany komórek i transformuje je. 
\begin{itemize}
\item Operacja analizy przeprowadzana jest przy użyciu 2 tablic dwuwymiarowych. Jedna z nich służy do przechowywania nowych stanów, a druga do badania sąsiedztwa i ustalania tych nowych stanów.
\end{itemize}
\item Program pokazuje wynik pojedynczej generacji.
\item Program zapisuje pojedynczą generację do pliku w przypadku błędu tworzenia pliku pojawi się komunikat i program zostanie zamknięty.
\item Program ponownie wykonuje operacje z punktów 3-5, aż nie wykona ilości generacji równej tej podanej w punkcie 2.
\item Program pyta czy użytkownik chce jeszcze raz wykonać proces (powrót do punktu 1) dla tych samych danych czy zakończyć.
\end{enumerate}







\end{enumerate}


 \item Testowanie\\
 \begin{enumerate}
 \item Ogólny przebieg testowania\\
 Do testów kodu użyjemy JUnit, a \textit{GUI} będziemy testować na bieżąco tworząc aplikację.
 \end{enumerate}


\end{enumerate}



\end{document}