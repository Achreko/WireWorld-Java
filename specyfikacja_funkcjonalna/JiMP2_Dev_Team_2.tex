\documentclass[11pt]{article}
\usepackage[polish]{babel}
\usepackage[T1]{fontenc}
\usepackage[utf8]{inputenc}
\usepackage{graphicx}
\usepackage{enumitem}

\begin{document}

\begin{huge}
\begin{center}
\textbf{Specyfikacja funkcjonalna}\\
\end{center}
\end{huge}

 \renewcommand{\labelenumii}{\Roman{enumii}}
 \begin{enumerate}
 
 
 
 
 
\item Ogólna funkcjonalność

\begin{enumerate}[label=\arabic{enumi}.\arabic*.]
 \item  Korzystanie z programu\\
 \\Program wykonany jest w formie aplikacji okienkowej. Znajduje się on w katalogu $automat komorkowy$ pod nazwą $Wireworld$. Występuje w formie pliku wykonywalnego o rozszerzeniu $jar.$ W tym samym folderze znajduje się plik $dane.txt$. Zawiera on dane na temat stanów wybranych komórek. Poniżej znajduje się przykładowy plik:
 Jeśli użytkownik chce zmienić dane lub wprowadzić nowe to musi edytować ten plik. \\
\item  Uruchamianie programu\\
\\Uruchomienie programu następuje przez klikniecie na plik „Wireworld”. Po tym program spyta ile generacji planszy użytkownik chce wykonać, a następnie wyświetli je po kolei.\\
\item Możliwosci programu\\
\\Program poza pokazaniem poszczególnych generacji planszy zapisuje również te generacje do folderu „generacje”. Przy każdym użyciu programu folder ten jest czyszczony i nowe generacje są zapisywane.\\
\end{enumerate}



\item Scenariusz działania programu\\


\begin{enumerate}[label=\arabic{enumi}.\arabic*.]
\item Scenariusz ogólny
\begin{enumerate}[label*=\arabic*.]
\item Program wczytuje dane z pliku 
\item Program pyta użytkownika o liczbę generacji
\item Program analizuje stany komórek i transformuje je.
\item Program pokazuje wynik pojedynczej generacji
\item Program zapisuje pojedynczą generację do pliku
\item Program ponownie wykonuje operacje z punktów 3-5, aż nie wykona ilości generacji równej tej 		podanej w punkcie 2.
\item Program pyta czy użytkownik chce jeszcze raz wykonać proces (powrót do punktu 1) dla tych 		samych danych czy zakończyć.\\
\end{enumerate}



\item Scenariusz szczegółowy
\begin{enumerate}[label*=\arabic*.]
\item Program wczytuje dane z pliku. 
\begin{itemize}
\item Program sprawdza czy sposób podania danych jest poprawny, jeśli nie pokaże komunikat „błąd danych” i zakończy prace.
\end{itemize}
\item Program pyta użytkownika o liczbę generacji. 
\begin{itemize}
\item Sprawdzane jest czy użytkownik podał wartość numeryczną większą od 0. Jeśli tak to przechodzi dalej, jeśli nie to wyświetla informację o podaniu złej wartości i pyta ponownie o liczbę generacji.
\end{itemize}
\item Program usuwa wszystkie pliki z katalogu „generacje”
\begin{itemize}
\item Jeśli katalog jest pusty to tylko idzie dalej, nic nie usuwa.
\end{itemize}
\item Program analizuje stany komórek i transformuje je. 
\begin{itemize}
\item Operacja analizy przeprowadzana jest przy użyciu 2 tablic dwuwymiarowych. Jedna z nich służy do przechowywania nowych stanów, a druga do badania sąsiedztwa i ustalania tych nowych stanów.
\end{itemize}
\item Program pokazuje wynik pojedynczej generacji.
\item Program zapisuje pojedynczą generację do pliku w przypadku błędu tworzenia pliku pojawi się komunikat i program zostanie zamknięty.
\item Program ponownie wykonuje operacje z punktów 3-5, aż nie wykona ilości generacji równej tej podanej w punkcie 2.
\item Program pyta czy użytkownik chce jeszcze raz wykonać proces (powrót do punktu 1) dla tych samych danych czy zakończyć.
\end{enumerate}







\end{enumerate}





\end{enumerate}



\end{document}